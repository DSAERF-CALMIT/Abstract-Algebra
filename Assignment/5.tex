\documentclass[12pt]{scrartcl}
\title{Abstract Algebra Assignments}
\nonstopmode
\usepackage{graphicx}	% Required for including pictures
\usepackage[figurename=Figure]{caption}
\usepackage{float}    		% For tables and other floats
\usepackage{amsmath}  	% For math
\usepackage{bbm}  		% For mathBBM
\usepackage{amssymb}  	% For more math
\usepackage{fullpage} 	% Set margins and place page numbers at bottom center
\usepackage{paralist} 	% paragraph spacing
\usepackage{listings} 	% For source code
\usepackage{enumitem} 	% useful for itemization
\usepackage{siunitx}  	% standardization of si units
\usepackage{tikz,bm} 	% Useful for drawing plots
\usepackage{fancyhdr}
\usepackage{setspace}
\usepackage{mathtools}



\renewcommand{\headrulewidth}{0pt}
\renewcommand\footrule{\hrule height1pt}

\pagestyle{fancy}
\fancyhf{}
\lfoot{Abstract Algebra Assignments \raisebox{0.5\depth}{\scalebox{0.8}\textcopyright}~Jinwei Zou}
\rfoot{Page \thepage}

\begin{document}

	\begin{center}
		\hrule
		\vspace{15pt}
		{\textbf { \large Abstract Algebra Assignments \raisebox{0.5\depth}{\scalebox{0.8}\textcopyright}~BinaryPhi}}
	\end{center}

	\thispagestyle{empty}
	\indent {\textbf{Name:} \underline{\hbox to 60pt{}} \hspace{\fill} \textbf{Assignment:} Number 5 \vspace{5pt} \\
	\indent {\textbf{Score:} \underline{\hbox to 60pt{}} \hspace{\fill} \textbf{Last Edit:} \today~PDT \vspace{8pt} \\
	\hrule
	\vspace{15pt}	

%%%%%%%%%%%%%%%%
	\paragraph*{Problem 1: Definitions}

	\begin{enumerate}[label=(\alph*)]
	
	%%%%%A
	\item \begin{spacing}{1.2}Given a group $G$ and an element $a$ from $G$, if $$G=\{a \circ a \circ \cdots \circ a \coloneqq a^n ~ | ~ n \in \mathbb{Z} \}$$ $G$ is called a \underline{\hbox to 80pt{}} and $a$ is the \underline{\hbox to 70pt{}}. $G$ is also denoted by $\langle a \rangle$. \end{spacing}

\vspace{-0.2em}
	%%%%%B
	\item \begin{spacing}{1.2} Prove that the subgroup of a cyclic group is also cyclic.
\vspace{0.2em}

	\framebox(35.5em, 30em){}
\end{spacing}


	%%%%%C
	\item \begin{spacing}{1.3}Prove that the subgroup of an 'integer-addition' group - $\{\mathbb{Z}; +\}$ - has the form of $m\mathbb{Z}, m \in \mathbb{N}$.
	
	\vspace{0.5em}
	\framebox(35.5em, 14em){}
\end{spacing}
	
%%%%%D
	\item \begin{spacing}{1.30} Let a group $G = \langle a \rangle.$ Prove that $G$ is isomorphic to $\{\mathbb{Z}; +\}$ if the order of $G$ is infinite, and $G$ isomorphic to $\{\mathbb{Z}/m\mathbb{Z}; +\}$, or written as $\{\mathbb{Z}_m; +\}$, if the order of $G$ is finite $m$.

	\vspace{0.5em}
	\framebox(35.5em, 25em){}	
\end{spacing}
	\vspace{2em}


%%%%%E
	\begin{spacing}{1.4}
	\item  Assume $G$ is a cyclic group of order $m$, $m_1$ is a positive integer factor of $m$. Prove that there exists a unique subgroup $G_1$ of order $m_1$.
	\vspace{1em}

	\framebox(35.5em, 46em){}
	\end{spacing}

\newpage
%%%%%F
\item \begin{spacing}{1.2} Prove that every cyclic group is abelian.

	\vspace{0.5em}
	\framebox(35.5em, 13em){}
\end{spacing}


%%%%%G
\item \begin{spacing}{1.3} Assume $S$ is a non-empty subset of group $G$. Let $S^{-1}$ be equal to $\{a^{-1} ~ | ~ a \in S\}$. Then, prove that: $$\{a_1 \cdots a_m ~ | ~ a_1, \cdots , a_m \in S \cup S^{-1}\} \text{~is a subgroup of~} G.$$	

	\vspace{0.2em}
	\framebox(35.5em, 25em){}

Additionally, this subgroup $H$ of group $G$ is called the subgroup \underline{\hbox to 90pt{}}, denoted by $\langle S \rangle.$ (Note that in general $"\langle"$ and $"\rangle"$ have nothing to do with cyclic.)
\end{spacing}

	\end{enumerate}

\newpage

%%%%%%%%%%%%%%%%
\paragraph*{Problem 2: }\begin{spacing}{1.4}Let $G$ be a group, $a, b \in G$ and the corresponding orders are $m$ and $n$. Prove the following statements:\end{spacing}

\vspace{-0.5em}
\begin{enumerate}[label=(\alph*)]
	\item The order of $a^k$ is $\cfrac{m}{\text{gcd}(m, k)}$.
	\item \begin{spacing}{1.5} Assume $\langle a \rangle \cap \langle b \rangle = \{e\}, ab = ba.$ The order of $ab$ is the least common multiple of $m$ and $n$: $\text{lcm}(m,n)$.\end{spacing}

	\vspace{0.5em} 
	\framebox(35.6em, 43em){}

\end{enumerate}


%%%%%%%%%%%%%%%%
\paragraph*{Problem 3: }\begin{spacing}{1.2}Find the elements and the number of elements of the following subgroups.


\begin{enumerate}[label=(\alph*)]
\item  The cyclic subgroup of $\{\mathbb{Z}_{20}; +\}$ generated by $25$.

\vspace{0.25em} 
\item The cyclic subgroup of $\{\mathbb{Z}_{63}; +\}$ generated by $49$.

\vspace{0.25em} 
\item The cyclic subgroup of group $\{\mathbb{C}; \cdot\}$ of non-zero complex numbers 

generated by $i ~(\sqrt{-1})$.

\vspace{0.3em} 
\item The cyclic subgroup of group $\{\mathbb{C}; \cdot\}$ of non-zero complex numbers 

generated by $\cfrac{\sqrt{2}+\sqrt{2}i}{2}$.

\vspace{0.3em} 
\framebox(35.6em, 38em){}


\end{enumerate}

\end{spacing}


%%%%%%%%%%%%%%%%
\paragraph*{Problem 4: }\begin{spacing}{1.5}Let $m, n$ be two prime numbers and $m \neq n$. Let $k$ be a positive integer. 

\vspace{0.5em}
Find the number of generators in the following situations.

\begin{enumerate}[label=(\alph*)]
\item How many generators are there in $\{\mathbb{Z}_{mn}; +\}$?

\item How many generators are there in $\{\mathbb{Z}_{m^k}; +\}$?

\vspace{1em}
\framebox(35.6em, 43em){}

\end{enumerate}
\end{spacing}


%%%%%%%%%%%%%%%%
\paragraph*{Problem 5: }Euler's Totient Function\\

\noindent Euler's Totient Function is defined as:
$$\phi(m) = \big|\hspace{2pt}\left\{n \in \mathbb{N} ~ | ~ n < m, \text{gcd}(m,n) = 1 \left(\text{HCF}(m,n) = 1\right)\right\}\hspace{2pt}\big|~,~ m\in \mathbb{N}.$$
Assuming multiplicative rule applies to this function, prove the following statements.

\begin{enumerate}[label=(\alph*)]
\item Let $p$ be a prime number. $\phi(p^x) = p^x-p^{x-1}.$

\vspace{0.4em}
\item Let $d = p_1^{x_1}p_2^{x_2} \cdots p_k^{x_k};$ $p_1, p_2, \cdots , p_k$ are prime numbers. We have: \vspace{-0.5em}$$\phi(d) = \prod_{i=1}^{k}(p_i^{x_i}-p_i^{x_i-1}).$$

\framebox(35.6em, 36em){}
\end{enumerate}

%%%%%%%%%%%%%%%%
\paragraph*{Problem 6: }Let $r$ and $s$ be positive integers. 

\vspace{1em}
\noindent Show that $\{mr + ns ~ | ~m, n \in \mathbb{Z}\}$ is a subgroup of $\mathbb{Z}$.\\

\setlength{\leftskip}{-12pt}\framebox(38.5em, 28em){}

%%%%%%%%%%%%%%%%
\setlength{\leftskip}{0pt}\paragraph*{Problem 7: } Let $x$ and $y$ be two elements of a group $G$. 

\vspace{1em}
\noindent Show that if $xy$ has finite order $n$, then $yx$ also has order $n$.\\

\setlength{\leftskip}{-12pt}\framebox(38.5em, 15em){}


\end{document}















