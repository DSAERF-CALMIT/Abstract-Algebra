\documentclass[12pt]{scrartcl}
\title{Abstract Algebra Assignments}
\nonstopmode
\usepackage{graphicx}	% Required for including pictures
\usepackage[figurename=Figure]{caption}
\usepackage{float}    		% For tables and other floats
\usepackage{amsmath}  	% For math
\usepackage{bbm}  		% For mathBBM
\usepackage{amssymb}  	% For more math
\usepackage{fullpage} 	% Set margins and place page numbers at bottom center
\usepackage{paralist} 	% paragraph spacing
\usepackage{listings} 	% For source code
\usepackage{enumitem} 	% useful for itemization
\usepackage{siunitx}  	% standardization of si units
\usepackage{tikz,bm} 	% Useful for drawing plots
\usepackage{fancyhdr}
\usepackage{setspace}
\usepackage{mathtools}



\renewcommand{\headrulewidth}{0pt}
\renewcommand\footrule{\hrule height1pt}

\pagestyle{fancy}
\fancyhf{}
\lfoot{Abstract Algebra Assignments \raisebox{0.5\depth}{\scalebox{0.8}\textcopyright}~Jinwei Zou}
\rfoot{Page \thepage}

\begin{document}

	\begin{center}
		\hrule
		\vspace{15pt}
		{\textbf { \large Solution - Abstract Algebra Assignments \raisebox{0.5\depth}{\scalebox{0.8}\textcopyright}~BinaryPhi}}
	\end{center}

	\thispagestyle{empty}
	\indent {\textbf{Name:} \underline{\hbox to 60pt{}} \hspace{\fill} \textbf{Assignment:} Number 5 \vspace{5pt} \\
	\indent {\textbf{Score:} \underline{\hbox to 60pt{}} \hspace{\fill} \textbf{Last Edit:} \today~PDT \vspace{8pt} \\
	\hrule
	\vspace{15pt}	

%%%%%%%%%%%%%%%%
	\paragraph*{Problem 1: Definitions}

	\begin{enumerate}[label=(\alph*)]
	
	%%%%%A
	\item \begin{spacing}{1.2}Given a group $G$ and an element $a$ from $G$, if $$G=\{a \circ a \circ \cdots \circ a \coloneqq a^n ~ | ~ n \in \mathbb{Z} \}$$ $G$ is called a \underline{\textbf{Cyclic Group}} and $a$ is the \underline{\textbf{Generator}\vphantom{p}}. $G$ is also denoted by $\langle a \rangle$. \end{spacing}

\vspace{-0.2em}
	%%%%%B
	\item \begin{spacing}{1.2} Prove that the subgroup of a cyclic group is also cyclic.

\vspace{0.5em}
	\fbox{\begin{minipage}{35em}
		\vspace{0.2em}
		Suppose $S$ is a subgroup of a cyclic group $G$, and $G=\langle a \rangle$. 

\vspace{0.3em}
To prove $S$ is cyclic $\Longrightarrow$

\vspace{0.3em}
To prove $S = \langle a^n \rangle, n = \text{min}\{m \in \mathbb{Z}^{+} ~ | ~ a^m \in S\} \Longrightarrow$


\begin{enumerate}[label=(\alph*)]
\item $\langle a^n \rangle \subseteq S$, because $a^n \in S$ and the closure property.

\item To prove $S \subseteq \langle a^n \rangle \Longrightarrow$

To prove $\forall a^m \in S, a^m \in \langle a^n \rangle \Longrightarrow$

To prove $\forall a^m \in S, \exists \hspace{0.2em} q \text{~ s.t.~} a^m = a^{qn} \Longrightarrow$

Suppose $m = qn + r$ where $r$ is the remainder: $0 \leq r < n$. We have:

\setlength{\leftskip}{50pt}$a^m = a^{qn+r} \Longrightarrow$

$a^r = a^{m-qn} \Longrightarrow$

$\because a^{m-qn} = a^m (a^{n})^{-q} \in S$

$\therefore a^r \in S$
\vspace{0.2em}

\setlength{\leftskip}{0pt}If $r \neq 0$, $\because 0 \leq r < n, n$ is not the minimum value, which is an arresting contradiction.

Thus, $r = 0 \Longrightarrow m = qn \Longrightarrow S \subseteq \langle a^n \rangle$.
\end{enumerate}
Therefore, $S = \langle a^n \rangle$.

	\vspace{0.2em}
 	\end{minipage}}\end{spacing}


	%%%%%C
	\item \begin{spacing}{1.3}Prove that the subgroup of an 'integer-addition' group - $\{\mathbb{Z}; +\}$ - has the form of $m\mathbb{Z}, m \in \mathbb{N}$.
	
	\vspace{0.5em}
	\fbox{\begin{minipage}{35em}
		\vspace{0.2em}
		Considering the generator of group $\{\mathbb{Z}; +\}$ as $1$: 
	\vspace{-0.5em}
\begin{align*}
\{\mathbb{Z}; +\} = \{1 \circ 1 \circ \cdots \circ 1 \} &\coloneqq \{1^n ~ | ~ n \in \mathbb{Z}\}\\
\text{also~} &\coloneqq \{n \cdot 1 ~ | ~ n \in \mathbb{Z}\}
\end{align*}
Suppose $S$ is a subgroup of $\{\mathbb{Z}; +\}$. Assume the generator in the subgroup $S$ is $1^m \text{~ or ~} m \cdot 1, ~m \in \mathbb{Z}^+$. We have $S = \langle m \cdot 1 \rangle = \{n \cdot m ~ | ~ n \in \mathbb{Z}\} = m\mathbb{Z}$. 

For $S = \{0\}$, $m = 0$, $S = 0\mathbb{Z}$.

Therefore, the subgroup $S$  of group $\{\mathbb{Z}; +\}$ has the form of $m\mathbb{Z}, m \in \mathbb{N}$.
	\vspace{0.2em}
 	\end{minipage}}\end{spacing}
	\vspace{2em}
	
%%%%%D
	\item \begin{spacing}{1.30} Let a group $G = \langle a \rangle.$ Prove that $G$ is isomorphic to $\{\mathbb{Z}; +\}$ if the order of $G$ is infinite, and $G$ isomorphic to $\{\mathbb{Z}/m\mathbb{Z}; +\}$, or written as $\{\mathbb{Z}_m; +\}$, if the order of $G$ is finite $m$.

	\vspace{0.5em}
	\fbox{\begin{minipage}{35em}
		\vspace{0.2em}
		Let $\phi: \{\mathbb{Z}; +\} \longrightarrow G$, $n \mapsto (a \circ \cdots \circ a) \coloneqq a^n$.

\vspace{0.3em}
$\forall p, q, \in \{\mathbb{Z}; +\},$ we have: \vspace{-0.5em}
$$\phi(p+q) = a^{p+q} = \underbrace{a \circ \cdots \circ a}_{p} \circ \underbrace{a \circ \cdots \circ a}_q = a^p \circ a^q = \phi(p) \circ \phi(q).$$
Thus, $\phi$ is a group homomorphism. In addition, every element in $G$ can be expressed as $a^n,$ so $\phi$ is an epimorphism.

Based on the theorem that \vspace{-0.8em}
$$f \text{~is an epimorphism from~} G_1 \text{~to~} G_2, G_1 / \text{ker} ~ f \cong G_2,$$
\vspace{-0.8em}we have: 
$$\{\mathbb{Z}; +\} / \text{~ker~} \phi \cong G \text{~and ker~} \phi \text{~has the form of m}\mathbb{Z}, m \in \mathbb{N}$$
When $m = 0$, which means ker $\phi = \{0\}$, we have $\{\mathbb{Z}; +\} \cong G \Longrightarrow G$ is infinite.

Similarly, when $m \neq 0$, which means ker $\phi = m\mathbb{Z}$ where $m \in \mathbb{Z}^+$, we have $\{\mathbb{Z}; +\} / m\mathbb{Z} \cong G \Longrightarrow G \cong \{\mathbb{Z}_m; +\}$. Now, $G$ is finite with order $m$.
	\vspace{0.2em}
 	\end{minipage}}\end{spacing}
	\vspace{2em}


%%%%%E
	\begin{spacing}{1.4}
	\item  Assume $G$ is a cyclic group of order $m$, $m_1$ is a positive integer factor of $m$. Prove that there exists a unique subgroup $G_1$ of order $m_1$.

	\vspace{0.8em}
	\fbox{\begin{minipage}{35em}
			\vspace{0.2em}
Let $G = \langle a \rangle$, we have:
\begin{align*} 
a^{\displaystyle m} &= e \\
&\Downarrow\\
a^{ \left( \displaystyle m_1\frac{m}{m_1} \right) } &= a^{ \left(\displaystyle\frac{m}{m_1} \right)\displaystyle{m_1} } = e
\end{align*}
For a positive integer $p < m_1$, which means $0 < (m/m_1)p < m$, we have: $(a^{m/m_1})^p \neq e$.
Thus $\langle a^{m/m_1} \rangle$ is the subgroup of order $m_1$.

Then, assume $G_1 = \langle a^p \rangle$ where $p = \text{~min}\{m \in \mathbb{Z}^+ ~ | ~ a^m \in G_1\}$. For any $a^k \in G_1$, we have $p | k$. Since $a^m \in G_1$, $p | m$. Then, we have $(a^p)^{m/p} = e$, which means $G_1$ has an order of $m/p$, which is also by definition $m_1$. Thus, $m/p = m_1 \Longrightarrow p = m/m_1 \Longrightarrow G_1 = \langle a^{m/m_1} \rangle.$ The uniqueness of this subgroup is therefore proven.\\

Second method to prove the first statement (there exists a subgroup of order $m_1$):
The finite cyclic group of order $m$ is isomorphic to $\{\mathbb{Z}_m; +\}$, which also means: 
$$G \cong \{\mathbb{Z}_m; +\} = \left\{\bar{0}, \bar{1}, \cdots, \overline{m-1}\right\}.$$

Since $m_1 | m, \cfrac{m}{m_1}$ is a positive integer. We have:
\begin{align*}
\forall u &\in \langle \overline{(\cfrac{m}{m_1})}\rangle \coloneqq \left\{\bar{0}, \overline{(\cfrac{m}{m_1})}, \overline{(2\cfrac{m}{m_1})}, \cdots, \overline{(m_1-1)
\cfrac{m}{m_1}} \right\}, \\
u &\in \{\mathbb{Z}_m; +\},
\end{align*}

which is isomorphic to the subgroup of $G$ of order $m_1.$

		\vspace{0.2em}
	 	\end{minipage}}
	\end{spacing}


\newpage
%%%%%F
\item \begin{spacing}{1.2} Prove that every cyclic group is abelian.

	\vspace{0.5em}
	\fbox{\begin{minipage}{35.4em}
		\vspace{0.2em}
		Let $a$ be the generator of a cyclic group $G$. We have: $$G=\langle a \rangle = \{a^n ~ | ~ n \in \mathbb{Z}\}$$
		For any two element $m$ and $n$ in $G$, suppose $m = a^p$ and $n = a^q$. Then, according to the associative law, we have: $$m \circ n = a^p \circ a^q = a^{p+q} = a^{q+p} = a^q \circ a^p = n \circ m.$$
		Thus, every cyclic group is abelian.
	\vspace{0.2em}
 	\end{minipage}}\end{spacing}
	\vspace{2em}
	
%%%%%G
\item \begin{spacing}{1.3} Assume $S$ is a non-empty subset of group $G$. Let $S^{-1}$ be equal to $\{a^{-1} ~ | ~ a \in S\}$. Then, prove that: $$\{a_1 \cdots a_m ~ | ~ a_1, \cdots , a_m \in S \cup S^{-1}\} \text{~is a subgroup of~} G.$$

\fbox{\begin{minipage}{35.4em}
		\vspace{0.2em}
		All \textbf{four} statements can be proven intuitively. 

I'll only prove the closure property here.

Let $H = \{a_1 \cdots a_m ~ | ~ a_1, \cdots , a_m \in S \cup S^{-1}\}$. 
\begin{align*}
\forall x \in H, ~&\exists ~x_1, \cdots, x_p \in S \cup S^{-1} \text{~s.t.~} x = x_1 \cdots x_p.\\ 
\forall y \in H, ~&\exists ~y_1, \cdots, y_q \in S \cup S^{-1} \text{~s.t.~} y = y_1 \cdots y_q.
\end{align*}
What we need to do is to prove that $xy \in H$.
Since
$$x_1, \cdots, x_p, y_1, \cdots, y_q \in S \cup S^{-1},$$

it is apparent that
$$xy = x_1 \cdots x_p y_1 \cdots y_q \in H,$$

meaning that $\forall x, y \in H, xy \in H.$

Therefore, the closure property is proven.
	\vspace{0.2em}
 	\end{minipage}}

	\vspace{0.4em}
Additionally, this subgroup $H$ of group $G$ is called the subgroup \underline{\textbf{generated by $S$}}, denoted by $\langle S \rangle.$ (Note that in general $"\langle"$ and $"\rangle"$ have nothing to do with cyclic.)
\end{spacing}
	\vspace{2em}

	\end{enumerate}

\newpage

%%%%%%%%%%%%%%%%
\paragraph*{Problem 2: }Prove:

\begin{enumerate}[label=(\alph*)]
	\item Assuming, we have 

\vspace{0.5em}
	\begin{spacing}{1.2}\fbox{\begin{minipage}{35em}
\vspace{0.3em} 
First prove : 

	\end{minipage}}\end{spacing}



	\item Assuming $f$ is a group homomorphism from group $G_1$ to group $G_2$, then $$f \text{~is monomorphism} \iff \text{ker} ~ f = \{e_1\}, \text{~where} ~ e_1 \text{~is the identity of} ~ G_1.$$
	\begin{spacing}{1.5}\fbox{\begin{minipage}{35em}
\vspace{0.3em} 
"$\Longrightarrow$: " 

\setlength{\leftskip}{36pt}$\because f(e_1) = e_2, ~\therefore \{e_1\} \subseteq \text{ker} ~ f.$

$\forall a \in \text{ker} ~ f, f(a) = e_2 = f(e_1). \because f ~\text{is injective}, ~ \therefore a = e_1.$

Thus, ker $f = \{e_1\}$.

\setlength{\leftskip}{0pt}"$\Longleftarrow$: " 

\setlength{\leftskip}{36pt}If $f(a) = f(b), a, b \in G_1$, then
\vspace{0.5em}

\setlength{\leftskip}{80pt}$f(ab^{-1}) = f(a) f(b)^{-1} = e_2 \Longrightarrow ab^{-1} \in \text{ker} ~ f. $
\vspace{0.5em}

\setlength{\leftskip}{36pt}$\because \text{ker} ~ f = \{e_1\}, ~ \therefore ab^{-1} = e \Longrightarrow a = b.$

Thus, $f$ is a monomorphism.
	\end{minipage}}\end{spacing}
\end{enumerate}

%%%%%%%%%%%%%%%%
\paragraph*{Problem 3: }\begin{spacing}{1.5}Define a binary operation $\circ$ in the integer set $\mathbb{Z}$ such that: \vspace{-1em}
$$a \circ b = a + b - a \times b, ~~\forall a, b \in \mathbb{Z}.\vspace{-1em}$$
Prove that $\{\mathbb{Z}, \circ\}$ is a monoid, and is isomorphic to a monoid of $\mathbb{Z}$ with respect to the operation multiplication "$\times$".

\vspace{1.5em}
\setlength{\leftskip}{-12pt}\fbox{\begin{minipage}{37.4em}
\vspace{0.3em}
$\{\mathbb{Z}, \circ\}$ is a monoid:

\vspace{0.5em}
\setlength{\leftskip}{30pt}Let $a, b, c \in \mathbb{Z}$, we have:\vspace{-1em}
\begin{align*}
a \circ b &= a + b - a \times b = b \circ a\\
e \circ a &= 0 \circ a = 0 + a - 0 \times a = a\\
(a \circ b) \circ c &= (a + b - a \times b) + c - (a + b - a \times b)c\\
&= a + b + c - a \times b - a \times c - b \times c + a \times b \times c\\
&= a \circ (b \circ c).
\end{align*}
\begin{spacing}{0.4}Thus, $\{ \mathbb{Z}, \circ \}$ is a commutative monoid.\end{spacing}

\vspace{2em}
\setlength{\leftskip}{0pt}$\{\mathbb{Z}, \circ\}$ and a monoid of $\mathbb{Z}$ with the operation multiplication are isomorphic.
\vspace{0.5em}

\setlength{\leftskip}{30pt}We need to find a map $f$ that satisfies $f(m) \circ f(n) = f(m \times n)$. For a map $f(a) = 1-a,$ we have:
\begin{align*}
f(m) \circ f(n) &= f(m) + f(n) - f(m) \times f(n)\\
&= 1 - m + 1 - n - (1 - m) \times (1 - n)\\
&= 1 - m \times n\\
&= f(m) \times f(n).
\end{align*}Thus, $\{\mathbb{Z}, \circ\}$ and a monoid $\{\mathbb{Z}, \times \}$ are isomorphic.
\vspace{0.5em}
	\end{minipage}}
\end{spacing}

\newpage
%%%%%%%%%%%%%%%%
\paragraph*{Problem 4: }\begin{spacing}{1.5}
Let $G$ be a group, prove the following statements:
\vspace{-1em}
$$m \longrightarrow m^{-1} \text{~is an automorphism of} ~ G \text{~if and only if} ~ G \text{~is an Abelian Group.}$$
\fbox{\begin{minipage}{37.4em}
\vspace{0.3em}
Suppose the map $m \longrightarrow m^{-1}$ is $\phi$. Since $G$ is a group, $\phi$ is a surjection (one-to-one correspondence). If $\phi$ is an automorphism, we have:
\begin{align*}
\phi(a) \phi(b) &= \phi(ab) = (ab)^{-1} = b^{-1} a^{-1} \\
&= \phi(b) \phi(a), \forall a, b \in G.
\end{align*}
Thus, $G$ is a commutative group.

\vspace{1em}
If $G$ is a commutative group, we have:
\begin{align*}
\phi(ab) &= (ab)^{-1} = b^{-1} a^{-1}\\
&= \phi(b) \phi(a) = \phi(a) \phi(b)
\end{align*}
Thus, $f$ is an automorphism.

\vspace{0.5em}
	\end{minipage}}\end{spacing}

%%%%%%%%%%%%%%%%
\paragraph*{Problem 5: }\begin{spacing}{1.5}
Assume $G$ is an abelian group, prove that \vspace{-1em}
$$\forall n \in \mathbb{Z}, m \longrightarrow m^n \text{~is an endomorphism of} ~ G$$
\fbox{\begin{minipage}{37.6em}
\vspace{0.3em}
What we are going to prove is $\forall n \in \mathbb{Z}, \forall a,b \in G, (ab)^n = a^nb^n$. 

By using Mathematical induction, we have for $n=1$ the equation holds, and assuming the equation holds for $n-1$, which means:
\begin{align*}
(ab)^n &= (ab)(ab)^{n-1} \\
&= aba^{n-1}b^{n-1} \\
&= a^nb^n
\end{align*}

Thus, $\forall n \in \mathbb{Z}, a \longrightarrow a^n$ is an endomorphism of $G$.
\vspace{0.5em}
	\end{minipage}}


\end{spacing}


%%%%%%%%%%%%%%%%
\paragraph*{Problem 6: }
\begin{spacing}{1.3}Let $\phi: G \longrightarrow H$ be a group homomorphism.

\vspace{1em}
\noindent Prove that $\phi(G)$ is abelian if and only if $\forall a, b \in G, aba^{-1}b^{-1} \in \text{ker} ~ \phi$.\end{spacing}

\vspace{1.5em}
\setlength{\leftskip}{-12pt}\fbox{\begin{minipage}{38em}
\vspace{0.3em}
Assume
\vspace{-2em}
\begin{spacing}{1.5}\begin{align*}
&\phi(a) = \alpha \in \phi(G)\\
&\phi(b) = \beta \in \phi(G)\\
&\forall a, b\in G.
\end{align*}\end{spacing}
$\forall \alpha, \beta \in \phi(G), \phi(G)$ is abelian
\vspace{-2em}
\begin{spacing}{1.5}\begin{align*}
&\text{if and only if} ~~ \alpha \beta = \beta \alpha\\
&\text{if and only if} ~~ (\beta \alpha)^{-1}(\alpha \beta) = (\beta \alpha)^{-1} (\beta \alpha) = e|_{\phi(G)}\\
&\text{if and only if} ~~ \alpha^{-1} \beta^{-1} \alpha \beta = e|_{\phi(G)}\\
&\text{if and only if} ~~ \phi(a)^{-1}\phi(b)^{-1}\phi(a)\phi(b) = e|_{\phi(G)}\\
&\text{if and only if} ~~ \phi(a^{-1} b^{-1} a b) = e|_{\phi(G)}\\
&\text{if and only if} ~~ a^{-1} b^{-1} a b \in \text{ker} ~ \phi\\
&\text{if and only if} ~~ aba^{-1}b^{-1} \in \text{ker} ~ \phi, \text{WLOG}.
\end{align*}\end{spacing}

Therefore, $\phi(G)$ is abelian if and only if $\forall a, b \in G, aba^{-1}b^{-1} \in \text{ker} ~ \phi$
\vspace{0.5em}

	\end{minipage}}

\newpage
%%%%%%%%%%%%%%%%
\setlength{\leftskip}{0pt}\paragraph*{Problem 7: }\begin{spacing}{1.5}
The map $\phi: \mathbb{Z} \longrightarrow \mathbb{Z}$ defined by $\phi(n) = n-1$ for $n \in \mathbb{Z}$ is bijective. Give the expression of the binary operation "*" on $\mathbb{Z}$ such that $\phi$ is isomorphic.
$$\{\mathbb{Z}, \times\} \longrightarrow \{\mathbb{Z}, *\}$$
\fbox{\begin{minipage}{37.6em}
\vspace{0.3em}
If the map $\phi$ is isomorphic, we have: 
\begin{align*}
\phi(m \times n) &= \phi(m) * \phi(n) = (m-1) * (n-1)\\
\text{WLOG}, m * n &= \phi(m+1) * \phi(n+1)\\
&=\phi((m+1) \times (n+1)) \\
&=\phi(m \times n + m + n + 1) \\
&=m \times n + m + n
\end{align*}
Therefore, we have $\forall m, n \in \mathbb{Z}, m * n = m \times n + m + n$.
\vspace{0.5em}
	\end{minipage}}
\end{spacing}


%%%%%%%%%%%%%%%%
\paragraph*{Problem 8: }\begin{spacing}{1.5}
The map $\phi: \mathbb{Q} \longrightarrow \mathbb{Q}$ defined by $\phi(n) = 2n+1$ for $n \in \mathbb{Q}$ is bijective. Give the expression of the binary operation "*" on $\mathbb{Q}$ such that $\phi$ is isomorphic.
$$\{\mathbb{Q}, *\} \longrightarrow \{\mathbb{Q}, +\}$$
\fbox{\begin{minipage}{37.6em}
\vspace{0.3em}
The map $\phi^{-1}$ is isomorphic because $\phi$ is isomorphic, we have: 
\begin{align*}
\phi(m + n) &= \phi(m) * \phi(n) = (2m+1) * (2n+1)\\
m * n &= \phi^{-1}(2m+1) * \phi^{-1}(2n+1) = \phi^{-1}((2m+1) + (2n+1))  \\
&= \phi^{-1}(2m + 2n + 2)\\
&= m+n+\frac{1}{2}
\end{align*}
Therefore, we have $\forall m, n \in \mathbb{Z}, m * n = m+n+\displaystyle\frac{1}{2}$.
\vspace{0.5em}
	\end{minipage}}
\end{spacing}



\end{document}
















