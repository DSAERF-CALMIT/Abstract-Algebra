\documentclass[12pt]{scrartcl}
\title{Abstract Algebra Assignments}
\nonstopmode
\usepackage{graphicx}	% Required for including pictures
\usepackage[figurename=Figure]{caption}
\usepackage{float}    		% For tables and other floats
\usepackage{amsmath}  	% For math
\usepackage{bbm}  		% For mathBBM
\usepackage{amssymb}  	% For more math
\usepackage{fullpage} 	% Set margins and place page numbers at bottom center
\usepackage{paralist} 	% paragraph spacing
\usepackage{listings} 	% For source code
\usepackage{enumitem} 	% useful for itemization
\usepackage{siunitx}  	% standardization of si units
\usepackage{tikz,bm} 	% Useful for drawing plots
\usepackage{fancyhdr}
\usepackage{setspace}


\renewcommand{\headrulewidth}{0pt}
\renewcommand\footrule{\hrule height1pt}

\pagestyle{fancy}
\fancyhf{}
\lfoot{Abstract Algebra Assignments \raisebox{0.5\depth}{\scalebox{0.8}\textcopyright}~Jinwei Zou}
\rfoot{Page \thepage}

\begin{document}

	\begin{center}
		\hrule
		\vspace{15pt}
		{\textbf { \large Solution - Abstract Algebra Assignments \raisebox{0.5\depth}{\scalebox{0.8}\textcopyright}~BinaryPhi}}
	\end{center}

	\thispagestyle{empty}
	\indent {\textbf{Name:} \underline{\hbox to 60pt{}} \hspace{\fill} \textbf{Assignment:} Number 2 \vspace{5pt} \\
	\indent {\textbf{Score:} \underline{\hbox to 60pt{}} \hspace{\fill} \textbf{Last Edit:} \today~PDT \vspace{8pt} \\
	\hrule
	\vspace{15pt}	

%%%%%%%%%%%%%%%%
	\paragraph*{Problem 1: Definitions}

	\begin{enumerate}[label=(\alph*)]
	
	%%%%%A
	\item Let "$\circ$" be the binary operation in the non-empty set $S$, and satisfies the following:
	$$(a \circ b) \circ c = a \circ (b \circ c), ~~\forall a, b, c \in S.$$
	Then, the algebraic system $\{S; \circ \}$ is called a \underline{\textbf{Semigroup}} ($S$ is a \underline{semigroup} for short) \\
	

	%%%%%B
	\item
	If two elements $e_1$ and $e_2$ in the semigroup satisfy:
	\begin{align*}
	&e_1 \circ a = a,\\
	&a \circ e_2 = a, ~~\forall a \in S
	\end{align*}
	$e_1$ is called the \underline{\textbf{Left Identity}} of $S$, and $e_2$ is called the \underline{\textbf{Right Identity}} of $S$.

	If an element $e$ in the semigroup satisfies:
	$$e \circ a = a \circ e = a, ~~\forall a \in S,$$
	$e$ is called the \underline{\textbf{Identity Element}} of $S$. \\
	\rule{0pt}{15pt}The semigroup that has \underline{Identity Element} is called a \underline{\textbf{Monoid}\vphantom{Indentity}}. \\



	%%%%%C
	\item 
	Assuming a monoid $\{S; \circ\}$ has the identity element $e$ and an element $a \in S$, if:
	\begin{align*}
	&a_1 \circ a = e,\\
	&a \circ a_2 = e, ~~\forall a_1, a_2 \in S
	\end{align*}	
	$a_1$ is called the \underline{\textbf{Left Inverse}\vphantom{g}} of $a$, and $a_2$ is called the \underline{\textbf{Right Inverse}} of $a$.

	If:
	$$a_3 \circ a = a \circ a_3 = e, ~~\forall a_3 \in S,$$
	$a_3$ is called the \underline{\textbf{Inverse Element}\vphantom{g}} of $a$, and denoted by $a_3 = a^{-1}.$ \\


	%%%%%	D
	\item If every element in monoid $\{S; \circ\}$ is invertible, then $S$ is called a \underline{\textbf{Group}}. \\

\newpage

	%%%%%E
	\item A group is a set $S$ with an operation "$\circ$" that satisfies the following:
		
		\begin{itemize}[leftmargin=100pt]
		\item[\textbf{Closure: }] \underline{$\forall a, b \in S$, we have $a \circ b \in S$}; 
		\item[\textbf{Associativity: }] \underline{$\forall a, b, c \in S$, we have $(a \circ b) \circ c = a \circ (b \circ c)$};
		\item[\textbf{Identity: }] \underline{$\forall a \in S, ~ \exists ~ e \in S$, so $e \circ a = a \circ e = a$};
		\item[\textbf{Invertibility: }] \underline{$\forall a \in S, ~ \exists ~ b \in S$, so $b \circ a = a \circ b = e$}; \\
		\end{itemize} 


	%%%%%F
	\item Unilateral definition of the previous definition. Prove that a semigroup $S$ is a group if it satisfies the following: 
		
		\begin{itemize}[leftmargin=50pt]
		\item $\forall a \in S, ~ \exists ~ b \in S$, so $b \circ a = e$;
		\item $\forall a \in S, ~ \exists ~ e \in S$, so $e \circ a = a$; \\
		\end{itemize} 
	
		\fbox{\begin{minipage}{35.6em}
		\textbf{Invertibility: } Assume $(a^{-1})^{-1}$ is a left inverse of $a^{-1}$: $(a^{-1})^{-1} \circ a^{-1} = e$, and $b \circ a = e$ could be rewritten as $a^{-1} \circ a = e$. Then, 
		\begin{align*}
		a \circ a^{-1} &= e \circ (a \circ a^{-1}) = ((a^{-1})^{-1} \circ a^{-1})(a \circ a^{-1})\\
		&= (a^{-1})^{-1} \circ e \circ a^{-1} = (a^{-1})^{-1} \circ a^{-1} = e
		\end{align*}	

		\textbf{Identity: } By using the inverse property and the semigroup, we have:
		\begin{align*}
		a \circ e &= a \circ (a^{-1} \circ a)\\
		&= (a \circ a^{-1}) \circ a = e \circ a = a
		\end{align*}	
		\end{minipage}}
		\vspace{\stretch{0.5}}

	%%%%%G
	\item Interesting Question: Does the previous conclusions still hold if the semigroup has a left inverse and a right identity:

		\begin{itemize}[leftmargin=50pt]
		\item $\forall a \in S, ~ \exists ~ a^{-1} \in S$, so $a^{-1} \circ a = e$;
		\item $\forall a \in S, ~ \exists ~ e \in S$, so $a \circ e = a$.\\
		\end{itemize} 
		\vspace{\stretch{0.5}}
		\fbox{\begin{minipage}{35.5em}
		\begin{spacing}{1.1}
		No: Assuming a semigroup with operation $a \circ b = a \cdot \sqrt{b^2} = a|b|$ with an identity element $e$. For any element $m$, $m \circ e = m$. However, for instance, for a nagetive $m$, we have $e \circ m = e|m| \neq m$. Thus, although the right identity exists in this scenario, the left identity doesn't exist. \\ \end{spacing}

		No: Let $G = \big\{ (\begin{smallmatrix} x & y\\ 0 & 0 \end{smallmatrix}) ~ \big| ~ x, y \in \mathbb{Q}, x \neq 0 \big\}$. 

\rule{0pt}{16pt}Because $(\begin{smallmatrix} x & y\\ 0 & 0 \end{smallmatrix}) (\begin{smallmatrix} x_1 & y_1\\ 0 & 0 \end{smallmatrix}) = (\begin{smallmatrix} xx_1 & xy_1\\ 0 & 0 \end{smallmatrix}$), $G$ is a semigroup with a left identity $e = ( \begin{smallmatrix} 1 & 0\\ 0 & 0 \end{smallmatrix} )$. 

\rule{0pt}{16pt}Because $(\begin{smallmatrix} x & y\\ 0 & 0 \end{smallmatrix}) (\begin{smallmatrix} x^{-1} & 0\\ 0 & 0 \end{smallmatrix}) = (\begin{smallmatrix} 1 & 0\\ 0 & 0 \end{smallmatrix}) = e$, $(\begin{smallmatrix} x & y\\ 0 & 0 \end{smallmatrix})$ has a right inverse. 

\rule{0pt}{16pt}However, for $y \neq 0$, $(\begin{smallmatrix} x_1 & y_1\\ 0 & 0 \end{smallmatrix}) (\begin{smallmatrix} x & y\\ 0 & 0 \end{smallmatrix}) \neq e$ ($x_1x=1$ and $x_1y=0$ contradict).
		\end{minipage}}
		\vspace{\stretch{1}}


	%%%%%H
	\item Let the operation "$\circ$" in an algebraic system be commutative, the group $\{S; \circ\}$ is called the \rule{0pt}{15pt}\underline{\textbf{Abelian Group}} or \underline{\textbf{Commutative Group}}. \\


	%%%%%I
	\item Prove that the operation "$\circ$" in group $\mathbb{G}$ is left(right) \textbf{Cancellative}: 
	\begin{align*}
	\forall a, b, c \in \mathbb{G}, ~ &a \circ b = a \circ c ~ \Longrightarrow ~ b = c \\
	&b \circ a = c \circ a ~ \Longrightarrow ~ b = c.\\
	\end{align*}
	\fbox{\begin{minipage}{35.6em}
	Since $\mathbb{G}$ is a group, we have $a^{-1} \in \mathbb{G}$. By multiplying $a^{-1}$ to the left of both sides of $a \circ b =a \circ c$, we have:
	\begin{align*}
	a^{-1} \circ (a \circ b) &= a^{-1} \circ (a \circ c) \\
	(a^{-1} \circ a) \circ b &= (a^{-1} \circ a) \circ c \\
	\therefore b &= c.
	\end{align*}
	It is the same for the proof of right cancellation law.
	\end{minipage}}\\
	\vspace{\stretch{1}}


	%%%%%J
	\item The \rule{0pt}{20pt}number of elements in group $\mathbb{G}$ is called the \underline{\textbf{Order}} of $\mathbb{G}$, denoted by $|\mathbb{G}|$. 

	\begin{spacing}{1.2}If $|\mathbb{G}|$ is finite, we call $\mathbb{G}$ a \underline{\textbf{Finite Group}}.  If $|\mathbb{G}|$ has infinite order, we call $\mathbb{G}$ a \underline{\textbf{Infinite Group}}.\end{spacing}
	\vspace{\stretch{1}}


	%%%%%K
	\item \begin{spacing}{1.2}Assuming the group $\mathbb{G}$ has an operation (multiplication or addtion) and $a$ is an element of $\mathbb{G}$, if $\forall k \in \mathbb{N}$, $a^{k} \neq 1 (\neq e)$ or $ka \neq 0 (\neq e)$, we call the order of element $a$ is \underline{\textbf{Infinite}\vphantom{y}}. If $\exists ~ k \in N$, $a^{k} = e$ or $ka = 0$, the order of element $a$ is \underline{min$\{k \in \mathbb{N} ~ | ~ a^{k} = e (ka = 0)\}$}. \end{spacing}
	\vspace{\stretch{10}}


	\end{enumerate}


\newpage

%%%%%%%%%%%%%%%%
\paragraph*{Problem 2: Prove:}

	\begin{enumerate}[label=\arabic*)]
	
	\item There is only one inverse element of any element $a$ in group $\mathbb{G}$.

	\vspace{\stretch{0.05}}
	\fbox{\begin{minipage}{35em}
	\vspace{0.2em}
	Assuming $a_1$ and $a_2$ are two inverse elements of element $a$, we have 
	$$a_1 \circ a = e = a_2 \circ a.$$ 
	According to the right cancellation law, $a_1 = a_2$.
	\end{minipage}}
	\vspace{\stretch{0.1}}


	\item For a group $\mathbb{G}$, $\forall a, b \in \mathbb{G}$, equations $a \circ x = b$ and $x \circ a = b$ have one and only one solution.

	\vspace{\stretch{0.05}}
	\fbox{\begin{minipage}{35em}
	\vspace{0.2em}
	Since $\mathbb{G}$ is a group, we have $a^{-1} \in \mathbb{G}$. \\

	\begin{spacing}{1.2}Due to the closure property of group, we have $a^{-1} \circ b \in \mathbb{G}$, which is the(a) solution of $a \circ x = b$. \\ \end{spacing}

	\begin{spacing}{1.2}If $x_1$ and $x_2$ are both the solutions of $a \circ x = b$, we have $a \circ x_1 = b$ and $a \circ x_2 = b$, thus $a \circ x_1 = a \circ x_2$. \\ \end{spacing}

	According to the right cancellation law, $x_1 = x_2$.
	\end{minipage}}
	\vspace{\stretch{0.1}}

	\item If $\forall a, b \in S$ for which $S$ is a semigroup, $S$ is a group if $a \circ x = b, ~ x \circ a = b$ both have solutions.

	\vspace{\stretch{0.05}}
	\fbox{\begin{minipage}{35em}
	\vspace{0.2em}
	\textbf{Clossure: } \\ \rule{0pt}{16pt}Satisfied because $S$ is a semigroup.\\

	\textbf{Associativity: } \\ \rule{0pt}{16pt}Satisfied because $S$ is a semigroup.	\\

	\textbf{Identity: } \\
		\rule{0pt}{16pt}Since $x \circ a = a$ has solution in $S$, denoted by $e_a \circ a = a$. \\
		\rule{0pt}{16pt}$\forall c \in S$, $a \circ x = c$ has a solution denoted by $d$, which means:
		\begin{align*}
		a \circ d &= c \\
		e_a \circ (a \circ d) = (e_a \circ a) \circ d = a \circ d &= \underline{c = e_a \circ c}
		\end{align*}
	\textbf{Invertibility: } \\
		\rule{0pt}{16pt}Since $x \circ a = e_a$ has solution in $S$, the solution is the left inverse of $a$.
	\end{minipage}}
	\vspace{\stretch{0.08}}

	\end{enumerate}

\newpage

%%%%%%%%%%%%%%%%
\paragraph*{Problem 3:} Check if the following options are semigroups, monoids, or groups?

	\begin{enumerate}[label=\arabic*)]
	
	\item In $\mathbb{Z}$, $a \circ b = a - b$;

	\fbox{\begin{minipage}{35em}
	\vspace{1em}
		Association Law Fails. Not a semigroup.
	\vspace{1em}
	\end{minipage}}

	\item In $\mathbb{Z}$, $a \circ b = a + b + ab$;

	\fbox{\begin{minipage}{35em}
	\vspace{0.3em}
		\underline{Association Law:\vphantom{y}} 
		\begin{align*}
		(a \circ b) \circ c &= (a + b + ab) + c + (a + b + ab)c = a+b+c+ab+ac+bc+abc;\\
		a \circ (b \circ c) &= a + (b+c+bc) + a(b+c+bc) = a+b+c+ab+ac+bc+abc;\\
		\therefore (a \circ b) \circ c &= a \circ (b \circ c)
		\end{align*}
		\setlength\parindent{24pt}
		\indent Thus, the binary operation has associative property.\\

		\noindent \underline{Identity Element:}
		$$e \circ b = e + b + eb \Longrightarrow e = 0, ~ 0 \circ b = 0 + b + 0b = b;$$
		\indent Thus, for any element in $\mathbb{Z}$, there exists an identity element $0$.\\

		\noindent \underline{Inverse Element:\vphantom{y}}
		$$i \circ b = i + b + ib \Longrightarrow i = -1, ~ (-1) \circ b = (-1) + b + (-1)b = -1;$$
		\indent Thus, for $i = -1$, the inverse element doesn't exist.\\

		\noindent Therefore, $\{G; \circ \}$ is a monoid (with commutative binary operation).
	\end{minipage}}

	\item In $\mathbb{Z}$, $a \circ b = a + b - ab$;

	\fbox{\begin{minipage}{35em}
	\vspace{0.3em}
		\underline{Association Law: \checkmark\vphantom{y}} 
		$$(a \circ b) \circ c = a+b+c-ab-ac-bc+abc = a \circ (b \circ c);$$
		\setlength\parindent{24pt}

		\noindent \underline{Identity Element: \checkmark}
		$$e \circ b = e + b - eb \Longrightarrow e = 0, ~ 0 \circ b = 0 + b - 0b = b;$$

		\noindent \underline{Inverse Element:\vphantom{y}}
		$$i \circ b = i + b - ib \Longrightarrow i = 1, ~ 1 \circ b = 1 + b - 1b = 1;$$
		\indent Thus, for $i = 1$, the inverse element doesn't exist.\\

		\noindent Therefore, $\{G; \circ \}$ is a monoid (with commutative binary operation).
	\end{minipage}}

	\end{enumerate}

	\newpage

%%%%%%%%%%%%%%%%
\paragraph*{Problem 4:}

	\begin{spacing}{1.2}Define operation "$\circ$" in $S=\{x ~ | ~ x \in \mathbb{R}, x \neq -1\}$: $a \circ b = a+ b+ ab$. Prove that $S$ is a group with respect to the operation "$\circ$". Then, solve equation $2 \circ x \circ 3 = 7$.\end{spacing}

	\vspace{2em}
	\fbox{\begin{minipage}{36.5em}
	\vspace{0.3em}
	\underline{Association Law:\vphantom{y}} 
	$$(a \circ b) \circ c = a+b+c+ab+ac+bc+abc =  a \circ (b \circ c)$$
	\setlength\parindent{20pt}
	\indent Thus, the binary operation has associative property.\\

	\noindent \underline{Identity Element:}
	$$e \circ b = e + b + eb \Longrightarrow e = 0, ~ 0 \circ b = 0 + b + 0b = b;$$
	\indent Thus, for any element in $\mathbb{Z}$, there exists an identity element $0$.\\

	\noindent \underline{Inverse Element:\vphantom{y}}\\
	
	\indent Because $a \neq -1$, we have: 
	
	\begin{align*}
	a \circ \cfrac{-a}{1+a} &= a + \cfrac{-a}{1+a} + a ~ \cfrac{-a}{1+a} = \cfrac{a(1+a) - a - a^2}{1+a} = \cfrac{a+a^2-a-a^2}{1+a} \\[1em]
	&= 0
	\end{align*}
	\indent Thus, the inverse always exists. \\

	\noindent Therefore, $\{G; \circ \}$ is a commutative group.\\

	\noindent In addition, $2 \circ x \circ 3 = 7 \Longrightarrow$
	\begin{align*}
	x &= \cfrac{-2}{1+2} \circ 7 \circ \cfrac{-3}{1+3} \\[1em]
	&= \cfrac{-2}{3} \circ 7 \circ \cfrac{-3}{4} \\[1em]
	&= \cfrac{-2}{3} + 7 + \cfrac{-3}{4} + \cfrac{-2}{3} \cdot 7 + \cfrac{-2}{3} \cdot \cfrac{-3}{4} + 7 \cdot \cfrac{-3}{4} + \cfrac{-2}{3} \cdot 7 \cdot \cfrac{-3}{4} \\[1em]
	&= \cfrac{1}{3}
	\end{align*}
	\vspace{0.5em}
	\end{minipage}}

	\newpage


%%%%%%%%%%%%%%%%
\paragraph*{Problem 5:} Prove: \\
\setlength\parindent{12pt}

\indent $\mathbb{G}$ is an Abelian Group if the order of every non-identity element is $2$. \\

\fbox{\begin{minipage}{36.5em}
\vspace{0.2em}
Assuming $e$ is the identity element, we have:
$$\forall a \in \mathbb{G}, ~ a^2 = e \Longrightarrow a^{-1} = a.$$

Thus, 
$$\forall a, b \in \mathbb{G}, ab = (ab)^{-1} = b^{-1}a^{-1} = ba.$$

Therefore, $\mathbb{G}$ is an Abelian Group (Commutative Group).
\vspace{0.2em}
\end{minipage}}


\vspace{1em}
%%%%%%%%%%%%%%%%
\paragraph*{Problem 6:} Assuming $M$ is a monoid, $m \in M$. Define another multiplication rule "$\circ$": $a \circ b = amb$. \\

\indent Prove that $M$ is a semigroup with respect to "$\circ$". \\ 

\indent When is $M$ a monoid with respect to "$\circ$"?\\

\fbox{\begin{minipage}{36.5em}
\vspace{0.2em}
Suppose $a,b,c \in M$, then we have: 
\begin{align*}
(a \circ b) \circ c &= (amb) \circ c = ambmc\\
a \circ (b \circ c) &= a \circ (bmc) = ambmc\\
\therefore (a \circ b) \circ c &= a \circ (b \circ c)
\end{align*}
Thus, $M$ is a semigroup. \\

Then, assuming $1$ is the identity element of $M$ and $e$ is the identity element of $\{M; \circ \}$, then we have:
\begin{align*}
e \circ 1 &= 1 = em1 = em \\
1 \circ e &= 1 = 1me = me 
\end{align*}
Thus, $m$ is invertible and $e = m^{-1}$. Then: 
\begin{align*}
e \circ b &= m^{-1}mb = b \\
b \circ e &= bmm^{-1} = b 
\end{align*}

Therefore, $\{M; \circ\}$ is a monoid when and only when $m$ is invertible.
\end{minipage}}

\newpage

%%%%%%%%%%%%%%%%
\paragraph*{Problem 7:} \begin{spacing}{1.2}Assuming $M$ is a monoid with an identity element $e$. It is said that the element $a$ of $M$ is invertible if there exists an element $a^{-1}$ that satisfies $a^{-1} a = a a^{-1} = e.$\\ \end{spacing}

Prove the following statements: 

\begin{enumerate}[label=\arabic*)]

\item If $a,b,c \in M$ and $ab=ca=e$, then $a$ is invertible and $a^{-1} = b= c$. 

	\fbox{\begin{minipage}{35em}
	\vspace{0.2em}
		We have: $$ab=ca=e \Longrightarrow c(ab) = c(e) = c = b = eb = (ca)b;$$
		So that, we have $$ab = ba = e \Longrightarrow a^{-1} = b = c.$$
	\end{minipage}}


\item If $a \in M$ is invertible then $b = a^{-1}$m, when and only when $aba=a, ~ ab^2a=e$.

	\fbox{\begin{minipage}{35em}
	\vspace{0.2em}
		$$ab^2a = e = (ab^2)a = a(b^2a)$$
		Thus, $a$ is invertible and $a^{-1} = ab^2 = b^2a$.\\

		Then, because $aba = a$, we have: 
		$$a^{-1}aba = a^{-1}a \Longrightarrow ba = e \Longrightarrow a^{-1} = b.$$

	\end{minipage}}


\item The sufficient prerequisite of $G$, the subset of $M$, being a group is that every element in $G$ is invertible and for all $g_1, g_2 \in G$, we have $g_1^{-1} g_2 \in G$.

	\fbox{\begin{minipage}{35em}
	\vspace{0.3em}
	\begin{spacing}{1.2}$\Longrightarrow:$ If $G$ is a group, then every element $g$ of $G$ is invertible and every inverse of the element $g^{-1} \in G$. Then, we have: $\forall g_1, g_2 \in G \Longrightarrow g_1^{-1}g_2 \in G.$\\

	$\Longleftarrow:$ If $g \in G$ and $g$ is invertible and $g_1, g_2 \in G$, $g_1^{-1}g_2 \in G \Longrightarrow (g_1^{-1})^{-1}g_2 = g_1, g_2 \in G$. Additionally, when $g_1 = g_2 = g \Longrightarrow g_1^{-1}g_2 = e \in G$, and $g_1^{-1} = g_1^{-1}e \in G$.\end{spacing}\end{minipage}}


\item All invertible elements in $M$ is a group.

	\fbox{\begin{minipage}{35em}
	\vspace{0.3em}
	\begin{spacing}{1.2}Suppose the set of all invertible elements in $M$ is $U$. It is apparent that every element of $U$ is invertible. Assuming $g_1, g_2 \in G$, then: \end{spacing}
	$$(g_1^{-1}g_2)(g_2^{-1}g_1) = (g_2^{-1}g_1)(g_1^{-1}g_2) = e.$$

	\rule{0pt}{20pt}Therefore, $g_1^{-1}g_2 \in U \Longrightarrow U$ is a group.
	\end{minipage}}



\end{enumerate}


\end{document}
















