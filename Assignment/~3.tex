\documentclass[12pt]{scrartcl}
\title{Abstract Algebra Assignments}
\nonstopmode
\usepackage{graphicx}	% Required for including pictures
\usepackage[figurename=Figure]{caption}
\usepackage{float}    		% For tables and other floats
\usepackage{amsmath}  	% For math
\usepackage{bbm}  		% For mathBBM
\usepackage{amssymb}  	% For more math
\usepackage{fullpage} 	% Set margins and place page numbers at bottom center
\usepackage{paralist} 	% paragraph spacing
\usepackage{listings} 	% For source code
\usepackage{enumitem} 	% useful for itemization
\usepackage{siunitx}  	% standardization of si units
\usepackage{tikz,bm} 	% Useful for drawing plots
\usepackage{fancyhdr}
\usepackage{setspace}


\renewcommand{\headrulewidth}{0pt}
\renewcommand\footrule{\hrule height1pt}

\pagestyle{fancy}
\fancyhf{}
\lfoot{Abstract Algebra Assignments \raisebox{0.5\depth}{\scalebox{0.8}\textcopyright}~Jinwei Zou}
\rfoot{Page \thepage}

\begin{document}

	\begin{center}
		\hrule
		\vspace{15pt}
		{\textbf { \large Solution - Abstract Algebra Assignments \raisebox{0.5\depth}{\scalebox{0.8}\textcopyright}~BinaryPhi}}
	\end{center}

	\thispagestyle{empty}
	\indent {\textbf{Name:} \underline{\hbox to 60pt{}} \hspace{\fill} \textbf{Assignment:} Number 3 \vspace{5pt} \\
	\indent {\textbf{Score:} \underline{\hbox to 60pt{}} \hspace{\fill} \textbf{Last Edit:} \today~PDT \vspace{8pt} \\
	\hrule
	\vspace{15pt}	

%%%%%%%%%%%%%%%%
	\paragraph*{Problem 1: Definitions}

	\begin{enumerate}[label=(\alph*)]
	
	%%%%%A
	\item \begin{spacing}{1.2}Assuming $H$ is a non-empty subset of group $G$ while $H$ is also a group with respect to the operation of $G$, we call $H$ a \underline{\textbf{Subgroup}} of $G$. \end{spacing}

	%%%%%B
	\item \begin{spacing}{1.2}$H$ is a subgroup of a group $G$. If $H=\{e\}$ or $H = G$, $H$ is called a \underline{\textbf{Trivial Subgroup}}. Other subgroups are called the \underline{\textbf{Non-trivial Subgroup}}. \end{spacing}

	%%%%%C
	\item \begin{spacing}{1.2} Prove that the following statements are equivalent if $H$ is a non-empty subset of $G$. 
	\subitem 1. $H < G$.
	\subitem 2. $a, b \in H \Longrightarrow a \circ b \in H, a^{-1} \in H$.
	\subitem 3. $a, b \in H \Longrightarrow a \circ b^{-1} \in H$.

	\vspace{0.5em}
	\fbox{\begin{minipage}{35em}
		\vspace{0.2em}
		$1 \Rightarrow 2:$ \\
Since $H$ is a group, according to the closure property, we have $a \circ b \in H$. Any element $a$ must have an inverse element $a^{-1}$ in $H$. Because $H < G$, which means the operations in both groups are the same, indicating that the inverse of $a$ in $H$ is exactly the inverse of $a$ in $G$. Therefore, $a^{-1} \in H$.

		\vspace{0.6em}
		$2 \Rightarrow 3:$ \\
We have $b \in H \Longrightarrow b^{-1} \in H$, and $a, b^{-1} \in H$. Thus, $a \circ b^{-1} \in H$.

		\vspace{0.6em}
		$3 \Rightarrow 1:$ \\
We have $a, a \in H \Longrightarrow a \circ a^{-1} \in H$, which means $e \in H \Longrightarrow H$ has identity element in it. Then, we have $e, b \in H \Longrightarrow e \circ b^{-1} \in H$, which means $b^{-1} \in H \Longrightarrow$ every element in $H$ has its corresponding inverse element. Because $a, b^{-1} \in H \Longrightarrow a \circ (b^{-1})^{-1} \in H$, which means $a \circ b \in H$, indicating the closure property of the operation of $H$. Additionally, the operation of the elements in $H$ satisfies the associative law because $H$ is a subset of a group $G$. Therefore, $H$ is a group respect to the operation of $G$.

		\vspace{0.2em}
 	\end{minipage}}\end{spacing}

	%%%%%D
	\item Assume $H$ is a subgroup of group $G$, $a \in G$, then:
	$$a \circ H = \{a \circ h ~ | ~ h \in H\}, H \circ a = \{h \circ a ~ | ~ h \in H\}$$
	\begin{spacing}{1.2}(Or often written as:
	$aH = \{ah ~ | ~ h \in H\}, Ha = \{ha ~ | ~ h \in H\}$) are called the \underline{\textbf{left coset}\vphantom{y}} and \underline{\textbf{right coset}} of $H$ with the representative el ement $a$, respectively.\end{spacing}

	%%%%%E
	\item \begin{spacing}{1.2}Assuming $H$ is a subgroup of group $G$ and $aRb \iff a^{-1}b \in H,$\\ i) prove that the relation $R$ in $G$ is an equivalent relation and \\ii) the equivalent class of $a$, $\overline{a\vphantom{b}}$, is exactly the left coset of $H$ represented by $a$: $aH$; \\iii) thus the set of all left cosets of $H: \{aH\}$ is a partition of $G$.

	\vspace{0.5em}
	\fbox{\begin{minipage}{35em}
	\vspace{0.2em}
	For $a, b \in G$, we could determine that $a^{-1}b \in H$, thus $R$ is a relation in $G$. \\1) Reflexive Property: $\forall a \in G, a^{-1}a \in H \Longrightarrow e \in H,$ thus $aRa$. \\2) Symmetric Property: If $aRb$, then $a^{-1}b\in H$, thus $(a^{-1}b)^{-1} \in H$ because $H$ is a group. Therefore, $b^{-1}a \in H \Longrightarrow bRa.$  \\3) Transitive Property: If $aRb, bRc$; then $a^{-1}b \in H$ and $b^{-1}c \in H$. Since $H$ is a group, we have $a^{-1}bb^{-1}c \in H \Longrightarrow a^{-1}c \in H$, $aRb, bRc \Rightarrow aRc$.
Therefore, $R$ is an equivalent relation in $G$.	

	\vspace{0.5em}
	$\forall b \in \overline{a\vphantom{b}} ~ (b \in H),$ we have $aRb,$ thus $a^{-1}b \in H$. Assuming $h \in H$ that satisfies $a^{-1}b = h$, which is $b = ah \in aH,$ we have $\overline{a\vphantom{b}} \subseteq aH$ since $\forall b \in \overline{a\vphantom{b}}.$ Additionally, we have $\forall b \in aH$, then assuming $h \in H \Longrightarrow b=ah$. Thus, $a^{-1}b = h \in H \Longrightarrow b \in \overline{a\vphantom{b}}$. \\In conclusion, $\overline{a\vphantom{b}} \subseteq aH, aH \subseteq \overline{a\vphantom{b}} \Longrightarrow \overline{a\vphantom{b}} = aH$.

	\vspace{0.5em}
	An equivalent relation $R$ determines a partition of a set, each class is the equivalent class $\overline{a\vphantom{b}}$ with respect to this equivalent relation $R$. Since $\overline{a\vphantom{b}} = aH, \{aH\}$ is a partition of $G.$ 
	\vspace{0.2em}
	\end{minipage}}\end{spacing}

	%%%%%F
	\item \begin{spacing}{1.2}The quotient set $G/R$ of group $G$ with respect to the equivalent relation $aRb \iff a^{-1}b \in H, H < G$ is called the \underline{\textbf{Quotient Set of $G$ by left congruence modulo $H$}} or \underline{\textbf{Left Coset Space}}, denoted by $G/H^{\mathbb{L}}$.\end{spacing}

	%%%%%G
	\item \begin{spacing}{1.2}The \underline{\textbf{Index}\vphantom{y}} of a subgroup $H$ in a group $G$ is the number of left cosets or right cosets of $H$ in $G$, which is denoted by $[G:H]$ or $|G:H|$.
	\end{spacing}

\newpage
	%%%%%H
	\item \begin{spacing}{1.2}Assuming a group $G$ has a subgroup $H<G$, we define $H$ to be a \underline{\textbf{Normal Subgroup}} of $G$ (denoted by $H \lhd G$), if: $$ghg^{-1} \in H, \forall g \in G, \forall h \in H.$$\end{spacing}

\vspace{0.2em}
	%%%%%I
	\item Prove the following statements are equivalent assuming $G$ is a group and $H<G$:
	\subitem 1) $H \lhd G;$
	\subitem 2) $gH=Hg, \forall g \in G;$
	\subitem 3) $g_1H \cdot g_2H = g_1g_2H =\{g_1h_1g_2h_2 ~ | ~ h_1, h_2 \in H\}.$

	\vspace{1em}	\begin{spacing}{1.3}
	\setlength{\leftskip}{18pt}\fbox{\begin{minipage}{35em}
	\vspace{0.3em}
	$1) \Longrightarrow 2):$ Since $H \lhd G$, $\forall g \in G$ and $\forall h \in H$, we have: \begin{align*}gh &= ghg^{-1}g \in Hg;\\
			 	hg &= gg^{-1}hg \in gH;\\
				\text{Since}~ gh &\in gH ~\text{and}~ hg \in Hg\\
				\therefore~ gH&=Hg.\end{align*}
	$2) \Longrightarrow 3):$ $\forall g_1, g_2 \in G$, there is an element $g_1h_1g_2h_2$ in $g_1H \cdot g_2H$ where $h_1, h_2 \in H$. We have $h_1g_2 \in Hg_2 = g_2H$, and considering $h_3 \in H$ which satisfies $h_1g_2 = g_2h_3.$ Thus, \begin{align*}
g_1h_1g_2h_2=g_1g_2h_3h_2 &\in g_1g_2H;\\
g_1H \cdot g_2H &\subseteq g_1g_2H.
\end{align*}
Then, any element $g_1g_2h$ from $g_1g_2H$ has:
\begin{align*}
g_1g_2h &= g_1eg_2h \in g_1H \cdot g_2H;\\
g_1g_2H &\subseteq g_1H \cdot g_2H.\\
\therefore g_1g_2H &= g_1H \cdot g_2H.
\end{align*}

	$3) \Longrightarrow 1):$ $\forall g \in G, \forall h \in H,$ we have:
	$$ghg^{-1} = ghg^{-1}e \in gH \cdot g^{-1}H = gg^{-1}H = eH = H$$
	Therefore, $H \lhd G.$
	\vspace{0.2em}\end{minipage}}
	\end{spacing}
	
\newpage
	%%%%%J
	\item \begin{spacing}{1.2}Assuming $G$ is a group and $H<G$, $R$ is a relation defined by $aRb \iff a^{-1}b \in H$, then: $$R\text{~is a congruence relation in~}G \iff H \lhd G.$$

\fbox{\begin{minipage}{36em}
\vspace{0.3em}
$\Longleftarrow$: Assuming $a_1Rb_1, a_2Rb_2$, we have $a_1^{-1}~b_1 \in H, a_2^{-1}~b_2 \in H$. Since we have:
\begin{align*}(a_1~a_2)^{-1}~(b_1~b_2) &= a_2^{-1}~(a_1^{-1}~b_1)~a_2~a_2^{-1}~b_2;\\
\because H \lhd G, a_2^{-1}~(a_1^{-1}~b_1)~a_2 \in H &\Longrightarrow a_2^{-1}~(a_1^{-1}~b_1)~a_2~a_2^{-1}~b_2; \\ 
&\Longrightarrow (a_1~a_2)^{-1}~(b_1~b_2) \in H\end{align*}
\setlength{\leftskip}{30pt}Therefore, $(a_1~a_2)^{-1}R(b_1~b_2),$ which means $R$ is a congruence relation with respect to the operation in $G$.

\vspace{0.5em}
\setlength{\leftskip}{0pt}
$\Longrightarrow$: $\forall g \in G, \forall h \in H,$ in order to prove $ghg^{-1} \in H,$ we have:
\begin{align*}g^{-1}gh = h \in H &\Longrightarrow gR(gh),\\
gg^{-1}R(gh)g^{-1} &\Longrightarrow ~eRghg^{-1} \text{~because~} g^{-1}Rg^{-1},\\
\therefore e^{-1}ghg^{-1} &= ghg^{-1} \in H.
\end{align*}
\vspace{-15pt}
\end{minipage}}

\vspace{1em}
More importantly, the quotient set $G/R$ and the operation with respect to the congruence relation $R$ is, a group, which is also called the \underline{\textbf{Quotient Group}} or \underline{\textbf{Factor Group}} of $G$ by $H$, denoted by $G/H$.
\end{spacing}


	\end{enumerate}


\newpage

%%%%%%%%%%%%%%%%
\paragraph*{Problem 2: }Prove:

\begin{enumerate}[label=(\alph*)]
	\item Assuming $H$ is a non-empty and finite subset of group $G$, we have $$H < G \Longleftrightarrow H \text{~is closed under the operation of~} G$$

\vspace{0.5em}
	\fbox{\begin{minipage}{35em}
	\begin{spacing}{1.3} 	\vspace{0.3em} "$\Longrightarrow$": \\Since $G$ is a group, the operation in $G$ must have associative property, left and right cancellative properties. Thus, the elements in the subset $H$ with respect to the operation in $G$ also have associative and cancellative properties. Because $H$ is closed under the operation of $G$, $H$ is a finite semigroup which also has the cancellative property. *Thus $H$ is a group with respect to the operation of $G$.\end{spacing}$\therefore H < G$ \vspace{0.6em} 

	\textbf{*}: Why? Recall the 2nd lecture.
	\end{minipage}}

%\framebox(36em, 25em){}

\vspace{0.8em}
\item \rule{0pt}{20pt}If $H_1$ and $H_2$ are both subgroup of group $G$, then $H_1 \cap H_2 < G$.

\vspace{0.5em}
\fbox{\begin{minipage}{35em}
	\begin{spacing}{1.3} 	\vspace{0.3em} 
$e \in H_1 \cap H_2, \forall a, b \in H_1 \cap H_2, $we have $a, b \in H_1$ and $a, b \in H_2 \Longrightarrow a \circ b^{-1} \in H_1$ and $a \circ b^{-1} \in H_2$ because $H_1$ and $H_2$ are two subgroups of $G$.

\vspace{0.8em}
Thus, $a \circ b^{-1} \in H_1 \cap H_2 \Longrightarrow H_1 \cap H_2 < G$.\end{spacing} \vspace{0.3em} \end{minipage}}


\vspace{0.8em}
\item \rule{0pt}{20pt}$[\mathbb{Z}: m \circ \mathbb{Z}] = m,$ where $m \in \mathbb{N}$

\vspace{0.8em}
\fbox{\begin{minipage}{35em}
	\begin{spacing}{1.3} 	\vspace{0.3em} 
Considering the left coset space of $\mathbb{Z}$ modulo $m \circ \mathbb{Z}$, we have:
\begin{align*}
\mathbb{Z} &= (0+m \circ \mathbb{Z}) \cup (1+m \circ \mathbb{Z}) \cup \cdots \cup ((m-1)+m \circ \mathbb{Z})\\
&= \overline{0\vphantom{b}} \cup \overline{1\vphantom{b}} \cup \cdots \cup \overline{(m-1)\vphantom{b}}.\end{align*}$\therefore ~~ [\mathbb{Z}:m \circ \mathbb{Z}]=m$\end{spacing}\vspace{0.2em}\end{minipage}}


\end{enumerate}

\newpage
%%%%%%%%%%%%%%%%
\paragraph*{Problem 3: }\textbf{Lagrange Theorem:}\\

\vspace{0.6em}
\setlength{\leftskip}{12pt}\noindent For a finite group $G$, $H<G$, then we have: $$|G| = [G:H]\cdot|H|,$$ which means the order of the subgroup $H$ is a factor of the order of $G$.\\

\setlength{\leftskip}{0pt}
\vspace{1em}
\fbox{\begin{minipage}{36em}
\vspace{0.8em}
\begin{spacing}{1.3} First of all, the number of elements in any left coset $aH$ of $H$ is equal to the number of elements in $H$ (which is denoted by $|H|$). It will be easier to think the map $h \rightarrow ah, \forall h \in H$. \vspace{0.5em}\\Then, $G$ can be described by the union of all non-intersecting left cosets of $H$, which is $[G:H]$ of them.\vspace{0.5em}\\ Therefore, there are $[G:H] \cdot |H|$ elements in $G \Longrightarrow |G| = [G:H]\cdot|H|$.\end{spacing}
\vspace{0.3em}
\end{minipage}}

\vspace{2em}
%%%%%%%%%%%%%%%%
\paragraph*{Problem 4: }\textbf{Corollary} of Lagrange Theorem: \\

\setlength{\leftskip}{12pt}\noindent If $G$ is a finite group and $K<G, H<K,$ we have: $$[G:H] = [G:K] \cdot [K:H].$$

\setlength{\leftskip}{0pt}
\vspace{0.8em}
\fbox{\begin{minipage}{36em}
\vspace{0.3em}
\begin{spacing}{1.2} 
According to Lagrange Theorem, we have \begin{align*}
|G| &= [G:K] \cdot |K| = [G:K] \cdot [K:H] \cdot |H|,\\
|G| &= [G:H] \cdot |H|.\\
[G:H] \cdot |H| &= [G:K] \cdot [K:H] \cdot |H|,\\
\therefore [G:H] &= [G:K] \cdot [K:H].
\end{align*}
Therefore, the corollary is proved.
\end{spacing}
\vspace{0.2em}
\end{minipage}}


\newpage
%%%%%%%%%%%%%%%%
\paragraph*{Problem 5: }Which of the following are true?
\begin{enumerate}[label=(\alph*)]
\begin{spacing}{1.2}
\item \underline{False}~~There exists a group in which the cancellation law fails.
\item \underline{False}~~Every group has exactly two improper subgroups.
\item \underline{True}~~Every group is a subgroup of itself.
\item \underline{False}~~A subgroup can be defined as the subset of a group.
\item \underline{False}~~Every set of numbers that is a group under addition is also a group under multiplication.
\end{spacing}
\end{enumerate}


%%%%%%%%%%%%%%%%
\paragraph*{Problem 6: }\begin{spacing}{1.4}Prove that \\if $G$ is an abelian group, written multiplicatively, with identity element $e$, then all elements $x$ of $G$ satisfying the equation $x^2=e$ form a subgroup $H$ of $G$.

\vspace{1em}
\fbox{\begin{minipage}{36em}
\vspace{0.3em}
\textbf{Closure: }

\setlength{\leftskip}{20pt}$\forall a, b \in H$, since $G$ is abelian, we have $(ab)^2 = a^2b^2=ee=e$, so $ab \in H \Longrightarrow H$ is closed.

\setlength{\leftskip}{0pt}\textbf{Identity: }

\setlength{\leftskip}{20pt}$\because ~ee=e,$ we have $e \in H$.

\setlength{\leftskip}{0pt}\textbf{Inverses: }

\setlength{\leftskip}{20pt}$\because ~\forall a \in H, aa = e$, which means the element of $H$ and its inverse is the same.\\

\setlength{\leftskip}{0pt}(Ref: John B. Fraleigh, Victor J. Katz. A first course in abstract algebra, 2003.)\\

\end{minipage}}
\end{spacing}

\newpage
%%%%%%%%%%%%%%%%
\paragraph*{Problem 7: } \begin{spacing}{1.4}Assume $H, K$ are two normal subgroups of group $G$ and $H \cap K = \{1\}.$ Prove the following $$hk=kh, \forall h \in H, \forall k \in K.$$

\vspace{0.8em}
\fbox{\begin{minipage}{36em}
\vspace{0.3em}
Because $H, K$ are normal subgroups, we have:
\begin{align*}
hkh^{-1} \in K; ~~&kh^{-1}k^{-1} \in H;\\
hkh^{-1}k^{-1} =~& (hkh^{-1})k^{-1} \in K\\
=~& h(kh^{-1}k^{-1}) \in H\\
\in~& K \cap H = \{1\}
\end{align*}
Therefore, $hkh^{-1}k^{-1} = 1 \Longrightarrow hk = kh.$\vspace{0.3em}
\end{minipage}}\end{spacing} 

%%%%%%%%%%%%%%%%
\paragraph*{Problem 8: } \begin{spacing}{1.4}Assume $H$ is a normal subgroup of group $G$. Prove that the sufficient prerequisite for $G/H$ to be an abelian group is the following: $$gkg^{-1}k^{-1} \in H, \forall g, k \in G.$$

\vspace{0.8em}
\fbox{\begin{minipage}{36em}
\vspace{0.3em}
The quotient group $G/H = \{gH ~ | ~ g \in G\}$ has $gHkH = gkH, (gH)^{-1} = g^{-1}H.$ \\Then, \begin{align*}gHkH = kHgH &\text{~if and only if~} gHkH(gH)^{-1}(kH)^{-1} = H\\
&\text{~if and only if~} gkg^{-1}k^{-1}H = H\\
&\text{~if and only if~} gkg^{-1}k^{-1} \in H\\
\end{align*}\end{minipage}}
\end{spacing}

\end{document}
















