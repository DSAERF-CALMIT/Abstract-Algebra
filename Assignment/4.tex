\documentclass[12pt]{scrartcl}
\title{Abstract Algebra Assignments}
\nonstopmode
\usepackage{graphicx}	% Required for including pictures
\usepackage[figurename=Figure]{caption}
\usepackage{float}    		% For tables and other floats
\usepackage{amsmath}  	% For math
\usepackage{bbm}  		% For mathBBM
\usepackage{amssymb}  	% For more math
\usepackage{fullpage} 	% Set margins and place page numbers at bottom center
\usepackage{paralist} 	% paragraph spacing
\usepackage{listings} 	% For source code
\usepackage{enumitem} 	% useful for itemization
\usepackage{siunitx}  	% standardization of si units
\usepackage{tikz,bm} 	% Useful for drawing plots
\usepackage{fancyhdr}
\usepackage{setspace}


\renewcommand{\headrulewidth}{0pt}
\renewcommand\footrule{\hrule height1pt}

\pagestyle{fancy}
\fancyhf{}
\lfoot{Abstract Algebra Assignments \raisebox{0.5\depth}{\scalebox{0.8}\textcopyright}~Jinwei Zou}
\rfoot{Page \thepage}

\begin{document}

	\begin{center}
		\hrule
		\vspace{15pt}
		{\textbf { \large Abstract Algebra Assignments \raisebox{0.5\depth}{\scalebox{0.8}\textcopyright}~BinaryPhi}}
	\end{center}

	\thispagestyle{empty}
	\indent {\textbf{Name:} \underline{\hbox to 60pt{}} \hspace{\fill} \textbf{Assignment:} Number 4 \vspace{5pt} \\
	\indent {\textbf{Score:} \underline{\hbox to 60pt{}} \hspace{\fill} \textbf{Last Edit:} \today~PDT \vspace{8pt} \\
	\hrule
	\vspace{15pt}	

%%%%%%%%%%%%%%%%
	\paragraph*{Problem 1: Definitions}

	\begin{enumerate}[label=(\alph*)]
	
	%%%%%A
	\item \begin{spacing}{1.2}Assuming $\{G_1; \circ\}$ and $\{G_2; *\}$ are two groups and $f$ is a map from $G_1$ to $G_2$, if: $$f(a \circ b) = f(a) * f(b), ~~\forall a, b \in G_1,$$ the map $f$ is called a \underline{\hbox to 120pt{}}.
\vspace{0.5em}

\setlength{\leftskip}{0pt}If $G_1$ and $G_2$ are two same groups,

\setlength{\leftskip}{24pt}$f$ is called an \underline{\hbox to 100pt{}}.
\vspace{0.5em}

\setlength{\leftskip}{0pt}If a \underline{\hbox to 120pt{}} $f$ is an injection (one-to-one),

\setlength{\leftskip}{24pt}$f$ is called a \underline{\hbox to 100pt{}}.
\vspace{0.5em}

\setlength{\leftskip}{0pt}If a \underline{\hbox to 120pt{}} $f$ is a surjection (onto), 

\setlength{\leftskip}{24pt}$f$ is called an \underline{\hbox to 100pt{}}.
\vspace{0.5em}

\setlength{\leftskip}{0pt}If a \underline{\hbox to 120pt{}} $f$ is a bijection (one-to-one correspondence, invertible), 

\setlength{\leftskip}{24pt}$f$ is called an \underline{\hbox to 100pt{}}, and $G_1$ and $G_2$ are \underline{\hbox to 100pt{}}, which is denoted by $G_1 \cong G_2.$
\end{spacing}

	%%%%%B
	\item Supplement:

\vspace{0.2em}
\underline{\hbox to 200pt{}} $$f: A \rightarrow B, \forall \hspace{0.2em} a, b \in A, \text{~such that }f(a) = f(b) \Longrightarrow a = b.$$

\underline{\hbox to 200pt{}} $$f: A \rightarrow B, \forall \hspace{0.2em} b \in B, \exists \hspace{0.2em} a \in A \text{~ s.t.} ~ f(a) = b.$$

\underline{\hbox to 200pt{}} $$f: A \rightarrow B, \forall \hspace{0.2em} b \in B, \text{exists a unique} ~ a \in A \text{~ s.t.} ~ f(a) = b.$$

	%%%%%C
	\item \begin{spacing}{1.3}Assuming $f$ is a group homomorphism from group $G_1$ to group $G_2$, then the set of all elements from $G_1$ which map to element $e$ in $G_2$ is called the \underline{\hbox to 50pt{}} of group homomorphism $f$, which is denoted by \underline{\hbox to 30pt{}}. Mathematically written as: $$ \underline{\hbox to 30pt{}} := \{g_1 \in G_1 ~ | ~ f(g_1) = e\}.$$
\end{spacing}
	


	%%%%%D
	\item \begin{spacing}{1.2} Assuming $f$ is a group homomorphism from group $G_1$ to group $G_2$, $e_1, e_2$ are the identity elements in $G_1, G_2$ respectively, $\circ, *$ are the operations in $G_1, G_2$ respectively, prove that $f(e_1) = e_2$ and $\forall a \in G_1, f(a^{-1}) = f(a)^{-1}.$

\vspace{0.5em}
\end{spacing}
\framebox(35.6em, 15em){}
	\vspace{2em}


%%%%%E
	\item \begin{spacing}{1.3}Assuming $f$ is a group homomorphism from group $G_1$ to group $G_2$, $H < G_1$, prove that the image set of $H$, $f(H)$ is a subgroup of $G_2$.\end{spacing}

	\vspace{0.5em}
\framebox(35.6em, 12em){}

\newpage
%%%%%F
	\begin{spacing}{1.3}
	\item Assuming $G$ is a group, $H \lhd G$, $\iota$ is a map from $G$ to $G/H$: \vspace{-1em}$$\iota(a) = aH, ~ \forall a \in G.\vspace{-1em}$$ Then, $\iota$ is an epimorphism, and is called the \underline{\hbox to 150pt{}} from group $G$ to quotient group $G/H$.
\end{spacing}

%%%%%G 
\vspace{-0.1em}
	\item \begin{spacing}{1.2}\textbf{Group Isomorphism Theorem I $|$ Fundamental Theorem on Group Homomorphisms}\end{spacing}

Prove that if $f$ is an epimorphism from group $G_1$ to group $G_2$, $G_1/\text{ker}~ f \cong G_2.$

\vspace{0.5em}
	\framebox(35.6em, 40em){}

%%%%%H
	\item \textbf{Group Isomorphism Theorem II}
\begin{spacing}{1.3}Let $G$ be a group, $N \lhd G$, and $H$ is a subgroup of $G$. Then:
	
\setlength{\leftskip}{20pt}1. $HN$ is a subgroup of $G$ which contains $N.$

2. $(H \cap N) \lhd H.$

3. $HN/N \cong H/(H \cap N).$

\end{spacing}


%%%%%I
\item \textbf{Group Isomorphism Theorem III}
\begin{spacing}{1.3}
Let $G$ be a group, $N \lhd G, N \lhd G, N \subseteq H$. Then:

\setlength{\leftskip}{20pt}1. $H/N \lhd G/N$

2. $(G/N)/(H/N) \cong G/H$
\end{spacing}


%%%%%J
\item \begin{spacing}{1.3}\textbf{Group Isomorphism Theorem IV $|$ Correspondence Theorem}

Assume $f$ is an epimorphism from group $G_1$ to $G_2$, and the kernel of group homomorphism $f$ is $F = \text{ker} ~ f$. We have: 

\setlength{\leftskip}{20pt}1. The map from a subgroup of $G_1$ that contains $N$ to a subgroup of $G_2$ is bijective.

2. The bijection from the subgroup of $G_1$ that contains $N$ to the subgroup of $G_2$ is also a map from a normal subgroup onto a normal subgroup.

3. For a normal subgroup $H \lhd G_1$ such that $H$ contains $N$, $G_1/H \cong G_2/f(H).$
\end{spacing}

\vspace{-0.2em}

	\end{enumerate}


\newpage

%%%%%%%%%%%%%%%%
\paragraph*{Problem 2: }Prove:

\begin{enumerate}[label=(\alph*)]
	\item Assuming $f$ is a group homomorphism from group $G_1$ to $G_2$, we have ker $f \lhd G_1$.

\vspace{0.5em}
	\framebox(35.6em, 22em){}

\vspace{0.5em}

	\item Assuming $f$ is a group homomorphism from group $G_1$ to group $G_2$, then $$f \text{~is monomorphism} \iff \text{ker} ~ f = \{e_1\}, \text{~where} ~ e_1 \text{~is the identity of} ~ G_1.$$
		\framebox(35.6em, 22em){}
\end{enumerate}

%%%%%%%%%%%%%%%%
\paragraph*{Problem 3: }\begin{spacing}{1.5}Define a binary operation $\circ$ in the integer set $\mathbb{Z}$ such that: \vspace{-1em}
$$a \circ b = a + b - a \times b, ~~\forall a, b \in \mathbb{Z}.\vspace{-1em}$$
Prove that $\{\mathbb{Z}, \circ\}$ is a monoid, and is isomorphic to a monoid of $\mathbb{Z}$ with respect to the operation multiplication "$\times$".\end{spacing}

\vspace{1.5em}
\setlength{\leftskip}{-12pt}\framebox(38em, 44em){}

\newpage
%%%%%%%%%%%%%%%%
\setlength{\leftskip}{0pt}\paragraph*{Problem 4: }Let $G$ be a group, prove the following statements:
$$m \longrightarrow m^{-1} \text{~is an automorphism of} ~ G \text{~if and only if} ~ G \text{~is an Abelian Group.}$$
	\framebox(38em, 26em){}

\vspace{1.5em}
%%%%%%%%%%%%%%%%
\paragraph*{Problem 5: }Assume $G$ is an abelian group, prove that
$$\forall n \in \mathbb{Z}, m \longrightarrow m^n \text{~is an endomorphism of} ~ G$$
	\framebox(38em, 16em){}



%%%%%%%%%%%%%%%%
\paragraph*{Problem 6: }
\begin{spacing}{1.3}Let $\phi: G \longrightarrow H$ be a group homomorphism.

\vspace{1em}
\noindent Prove that $\phi(G)$ is abelian if and only if $\forall a, b \in G, aba^{-1}b^{-1} \in \text{ker} ~ \phi$.\end{spacing}

\vspace{1.5em}
\setlength{\leftskip}{-12pt}\framebox(38em, 44em){}


\newpage
%%%%%%%%%%%%%%%%
\setlength{\leftskip}{0pt}\paragraph*{Problem 7: }\begin{spacing}{1.4}
The map $\phi: \mathbb{Z} \longrightarrow \mathbb{Z}$ defined by $\phi(n) = n-1$ for $n \in \mathbb{Z}$ is bijective. Give the expression of the binary operation "*" on $\mathbb{Z}$ such that $\phi$ is isomorphic.
$$\{\mathbb{Z}, \times\} \longrightarrow \{\mathbb{Z}, *\}$$
	\framebox(38em, 18em){}
\end{spacing}


%%%%%%%%%%%%%%%%
\paragraph*{Problem 8: }\begin{spacing}{1.4}
The map $\phi: \mathbb{Q} \longrightarrow \mathbb{Q}$ defined by $\phi(n) = 2n+1$ for $n \in \mathbb{Q}$ is bijective. Give the expression of the binary operation "*" on $\mathbb{Q}$ such that $\phi$ is isomorphic.
$$\{\mathbb{Q}, *\} \longrightarrow \{\mathbb{Q}, +\}$$
	\framebox(38em, 18em){}
\end{spacing}



\end{document}








